%% Generated by Sphinx.
\def\sphinxdocclass{report}
\documentclass[letterpaper,10pt,english]{sphinxmanual}
\ifdefined\pdfpxdimen
   \let\sphinxpxdimen\pdfpxdimen\else\newdimen\sphinxpxdimen
\fi \sphinxpxdimen=.75bp\relax

\PassOptionsToPackage{warn}{textcomp}
\usepackage[utf8]{inputenc}
\ifdefined\DeclareUnicodeCharacter
% support both utf8 and utf8x syntaxes
\edef\sphinxdqmaybe{\ifdefined\DeclareUnicodeCharacterAsOptional\string"\fi}
  \DeclareUnicodeCharacter{\sphinxdqmaybe00A0}{\nobreakspace}
  \DeclareUnicodeCharacter{\sphinxdqmaybe2500}{\sphinxunichar{2500}}
  \DeclareUnicodeCharacter{\sphinxdqmaybe2502}{\sphinxunichar{2502}}
  \DeclareUnicodeCharacter{\sphinxdqmaybe2514}{\sphinxunichar{2514}}
  \DeclareUnicodeCharacter{\sphinxdqmaybe251C}{\sphinxunichar{251C}}
  \DeclareUnicodeCharacter{\sphinxdqmaybe2572}{\textbackslash}
\fi
\usepackage{cmap}
\usepackage[T1]{fontenc}
\usepackage{amsmath,amssymb,amstext}
\usepackage{babel}
\usepackage{times}
\usepackage[Bjarne]{fncychap}
\usepackage{sphinx}

\fvset{fontsize=\small}
\usepackage{geometry}

% Include hyperref last.
\usepackage{hyperref}
% Fix anchor placement for figures with captions.
\usepackage{hypcap}% it must be loaded after hyperref.
% Set up styles of URL: it should be placed after hyperref.
\urlstyle{same}

\addto\captionsenglish{\renewcommand{\figurename}{Fig.}}
\addto\captionsenglish{\renewcommand{\tablename}{Table}}
\addto\captionsenglish{\renewcommand{\literalblockname}{Listing}}

\addto\captionsenglish{\renewcommand{\literalblockcontinuedname}{continued from previous page}}
\addto\captionsenglish{\renewcommand{\literalblockcontinuesname}{continues on next page}}
\addto\captionsenglish{\renewcommand{\sphinxnonalphabeticalgroupname}{Non-alphabetical}}
\addto\captionsenglish{\renewcommand{\sphinxsymbolsname}{Symbols}}
\addto\captionsenglish{\renewcommand{\sphinxnumbersname}{Numbers}}

\addto\extrasenglish{\def\pageautorefname{page}}

\setcounter{tocdepth}{2}



\title{BOPTEST Testcase 1}
\date{Nov 14, 2024}
\release{}
\author{}
\newcommand{\sphinxlogo}{\vbox{}}
\renewcommand{\releasename}{}
\makeindex
\begin{document}

\pagestyle{empty}
\maketitle
\pagestyle{plain}
\sphinxtableofcontents
\pagestyle{normal}
\phantomsection\label{\detokenize{index::doc}}



\chapter{Model Description}
\label{\detokenize{modelDescription:model-description}}\label{\detokenize{modelDescription::doc}}

Testcase 1 is a simple 1R1C network where a single zone is heated or cooled with a prescribed heat flow. This heat flow is controlled with a proportional feedback controller (P-controller) that activates when the zone temperature violates the prescribed lower or upper bound. The basic P-controller implemented in BOPTEST uses fixed lower and upper bounds of 20°C and 23°C respectively and has a gain of $2000$.\\

The heating / cooling power is not generated by a dedicated heating / cooling component but obtained by dividing the signal of the P-controller by a fixed efficiency of 99\% and taking the absolute value. Ambient temperature is assumed to have a prescribed sinusoidal profile and is separated by the zone with a fixed thermal resistance. This thermal resistance has a value of $0.01 K/W$ while the zonal capacitance has a value of $10^6 J/K$.\\

Figure \ref{\detokenize{modelicaWhole}} illustrates the testcase as implemented in Modelica.

\begin{figure}[htbp]
\centering
\capstart
\noindent\sphinxincludegraphics[scale=0.6]{{modelicaWhole}.png}
\caption{Figure 6 \textendash{} Modelica implementation of the 1R1C network.}\label{\detokenize{modelicaWhole}}\end{figure}


\renewcommand{\indexname}{Index}
\printindex
\end{document}
